\documentclass{report}

\usepackage[utf8]{inputenc}
\usepackage[francais]{babel}
\title{TER, développement d'un émulateur de GameBoy}
\author{BERGER Mickaël \\ DIVET Joachim \\ VINARD Florian}
\date{17 Mai 2013}

\begin{document}
\maketitle

\tableofcontents
\chapter{Introduction}
\section{Cadre}
\section{Motivation du sujet}
\section{Définition du projet}

\chapter{Organisation du projet}
\section{Choix des outils}
\section{Documentation}
\section{Etude de l'existant}
\section{Articles de recherches}

\chapter{Etude du Hardware}
\section{CPU}
\section{GPU}
\section{MMU}
\section{APU}
\section{Interruptions/Synchronisations}
\chapter{Réalisation}
\section{CPU}
\section{GPU}
\section{MMU}
\section{APU}
\section{Interruptions/Synchronisations}
\section{Fonctionnalités additionnelles}
\chapter{Conclusion}
\section{Bilan}
\section{Fonctionnalités pouvant être rajoutées}
\section{Emulation aujourd'hui}

%annexes
\appendix
\chapter{Générer un son en temps réel et sans fichier avec la SDL}
\chapter{Charger et créer une police au format BMP sans recours à une librairie externe avec SDL}
\chapter{Méthodes de débuggage et roms de test}
\end{document}
